\subsection{Tracking the centerline of a lane with a RC car}

Context: Automatic Control Project Course EL2425, KTH Royal Institute of
Technology, Stockholm, Sweden.\\

Problem statement: suppose a remotely controlled vehicle chassis of Ackermann
steering, equipped with a laser rangefinder and a computing unit. Given a
straight path (something that simulates a road lane, e.g. a corridor), the goal
is for the vehicle to navigate the path while always staying in the middle of it.
The solution to the problem involves solving two distinct and independent
sub-problems, centered around the (a) translational error, and (b) the
rotational error, with respect to the middle line of the path. Furthermore, the
problem can be approached by two ways: through the design of a PID controller,
which is easier to setup, faster in execution, but potentially difficult to tune,
and through that of a MPC controller, which requires more work, is slower in
execution, but is more robust than the PID controller. Figure
\ref{fig:centerline_pid_error} shows the evolution of the one-dimensional
angular error of the vehicle under control by the PID controller.

\begin{figure}[H]\centering
  \scalebox{0.6}{% This file was created by matlab2tikz.
%
%The latest updates can be retrieved from
%  http://www.mathworks.com/matlabcentral/fileexchange/22022-matlab2tikz-matlab2tikz
%where you can also make suggestions and rate matlab2tikz.
%
\definecolor{mycolor1}{rgb}{0.00000,0.44700,0.74100}%
%
\begin{tikzpicture}

\begin{axis}[%
width=4.133in,
height=3.26in,
at={(0.693in,0.44in)},
scale only axis,
xmin=0,
xmax=500,
xmajorgrids,
xlabel={time [samples]},
ymin=-200,
ymax=200,
ymajorgrids,
ylabel={angular error [degrees]},
axis background/.style={fill=white}
]
\addplot [color=mycolor1,solid,forget plot]
  table[row sep=crcr]{%
1	41.4030631258649\\
2	45.9373388478912\\
3	38.6599191363098\\
4	39.9193445475576\\
5	40.4454774344183\\
6	45.4133472253353\\
7	40.9283410728869\\
8	41.3953518422919\\
9	45.4017802999759\\
10	43.620924587111\\
11	40.4193451306538\\
12	40.686053772459\\
13	41.1582076743091\\
14	39.4189170776383\\
15	39.1483527940466\\
16	38.9403359677792\\
17	45.1273535444086\\
18	38.1607757419544\\
19	41.6881927878616\\
20	40.8747821445102\\
21	44.6496364534795\\
22	39.6432102446977\\
23	40.1350658272087\\
24	38.3966358343733\\
25	38.8953523419055\\
26	42.1329234801941\\
27	39.412920171547\\
28	42.4043517827158\\
29	39.1089221119681\\
30	42.1436366319073\\
31	44.1337830013197\\
32	39.3953529250017\\
33	41.183052487415\\
34	42.9193446310402\\
35	37.4193450471712\\
36	38.9214880607646\\
37	37.3786405348125\\
38	45.3914940350859\\
39	45.656491714504\\
40	41.4069221827459\\
41	41.6612067104509\\
42	43.402635156332\\
43	39.414204247111\\
44	41.1599186366962\\
45	38.6620638157092\\
46	22.6706307044175\\
47	36.9334829556569\\
48	44.6603496051927\\
49	40.6920506785503\\
50	41.6633479747557\\
51	41.6209256698208\\
52	39.6552074719749\\
53	40.6646342992219\\
54	45.610210352688\\
55	43.924485514583\\
56	40.1547777531533\\
57	41.220314584432\\
58	38.4193462133636\\
59	38.9202039852006\\
60	41.6277831797475\\
61	40.9107772414362\\
62	45.4124900365945\\
63	36.6856256359609\\
64	43.4047798357313\\
65	43.4056370244722\\
66	38.6582047588281\\
67	44.4064919643108\\
68	45.1813393596083\\
69	42.6616347634664\\
70	41.4172050325414\\
71	38.9103480222283\\
72	39.8953500930033\\
73	41.4390513926017\\
74	47.3709238376906\\
75	40.6423508070547\\
76	45.1616342638529\\
77	38.9484741381753\\
78	40.4090622808583\\
79	22.6710592987879\\
80	40.3876393925266\\
81	47.627353044795\\
82	39.6307827155029\\
83	40.8816396544369\\
84	39.4030642085747\\
85	42.9103492719033\\
86	44.6582049257933\\
87	40.9133488076587\\
88	39.9077776221982\\
89	39.9064935466342\\
90	38.4223446664092\\
91	42.1500638399162\\
92	43.1329246463865\\
93	41.1474945225959\\
94	40.1479236583212\\
95	40.1367802046904\\
96	41.4047775033465\\
97	41.8584933095322\\
98	43.6663453450916\\
99	42.1560641611021\\
100	41.6710592583287\\
101	40.1479236583212\\
102	41.4064918808282\\
103	41.6526348230426\\
104	39.6740587940841\\
105	42.3820682905487\\
106	39.7053274001044\\
107	38.7117500268264\\
108	40.6453526751949\\
109	43.6753441193231\\
110	42.3837826680303\\
111	42.6170677791321\\
112	43.1886214239734\\
113	42.9339100094452\\
114	45.1496370365757\\
115	42.3889258004753\\
116	43.6500655892048\\
117	39.3799235276667\\
118	41.8812105187116\\
119	42.6620616502896\\
120	39.9184873588168\\
121	41.1629205048364\\
122	39.909918886503\\
123	46.6693438816198\\
124	40.4129179226448\\
125	42.9390497267957\\
126	38.9060654936187\\
127	41.6714895602464\\
128	41.4077793714868\\
129	39.8777828464581\\
130	45.6569220164217\\
131	44.9017797168797\\
132	40.4219166968763\\
133	46.8246282324\\
134	39.3344891093077\\
135	40.0823435970054\\
136	43.1179255509691\\
137	40.0986370132704\\
138	34.1573491524106\\
139	35.9287689589372\\
140	39.9232001893441\\
141	40.4728835684629\\
142	36.3880684447693\\
143	39.9437624738404\\
144	39.3743535083985\\
145	35.6037783129442\\
146	34.864495712655\\
147	33.6547770030919\\
148	38.8936379644238\\
149	32.1971907285491\\
150	34.2656998583091\\
151	29.3029322187021\\
152	32.7746866379683\\
153	26.5644873992959\\
154	26.1029885925511\\
155	24.3688138075686\\
156	20.6508840026895\\
157	26.4632896090057\\
158	23.9397840125907\\
159	17.4983231063582\\
160	19.746614767904\\
161	18.2055977326495\\
162	17.4816625674252\\
163	15.1897857612383\\
164	13.7256809249063\\
165	19.2179888619963\\
166	15.9538875614241\\
167	8.62437314326932\\
168	8.69662288430971\\
169	10.5961445962672\\
170	11.8418669404926\\
171	11.5144148650775\\
172	12.2434376674598\\
173	11.8747824758766\\
174	10.8280568393458\\
175	6.74570784082933\\
176	1.65213674253122\\
177	3.77109855869943\\
178	0.88862479321329\\
179	2.79563399332665\\
180	2.06155087454307\\
181	2.9363521119282\\
182	-1.65136258327749\\
183	6.23058052315567\\
184	2.59768870787734\\
185	5.64657458968384\\
186	1.44775656511011\\
187	0.929879989573813\\
188	0.246060899903015\\
189	0.131812045511734\\
190	0.468561862676864\\
191	-4.37382476002579\\
192	1.71585397647269\\
193	-0.818857345688518\\
194	-1.01350573084198\\
195	-3.12532829178023\\
196	-0.0186118002483486\\
197	-1.94778297485911\\
198	-1.67224055057951\\
199	1.98520259311385\\
200	0.776419245941114\\
201	-1.47523921861783\\
202	-4.40742801008179\\
203	-7.06464651233084\\
204	-7.50541569494469\\
205	-3.4490416923984\\
206	-8.42322834108067\\
207	-4.02650366431512\\
208	-0.705141739640388\\
209	0.12997716588402\\
210	-2.00362468052243\\
211	0.609056719469911\\
212	-4.50774897483729\\
213	-4.41956482935008\\
214	-5.93846998094033\\
215	-5.12231360154258\\
216	-4.51658254386823\\
217	-6.39197020928026\\
218	-1.52971381923009\\
219	-7.65890968238874\\
220	-5.79945129878165\\
221	-1.9952141563332\\
222	-1.97727445103475\\
223	-5.43909436015319\\
224	-4.31904408156135\\
225	-1.94676698422007\\
226	-2.46999383903231\\
227	-3.79642959589689\\
228	-9.00644880279796\\
229	-5.66843631766976\\
230	-4.54506059072578\\
231	-5.4050906902571\\
232	-4.80178520351175\\
233	-3.91579782381344\\
234	-5.51574103547178\\
235	-4.23837869368986\\
236	-6.58880857719721\\
237	-0.949245496861503\\
238	-2.46515059458053\\
239	1.72938949041695\\
240	-3.36425962144809\\
241	-2.25974332746298\\
242	-4.61658931992191\\
243	-3.42615202094239\\
244	-4.24344584028036\\
245	1.35013651328453\\
246	-4.93578583106459\\
247	3.26465006358332\\
248	-1.26263314555849\\
249	3.674614879597\\
250	2.64551283867212\\
251	1.85686099073178\\
252	1.59003591554553\\
253	1.55365811458263\\
254	1.28501418136475\\
255	0.730446683380786\\
256	2.70634613195825\\
257	3.48954129908199\\
258	-0.0522938125907353\\
259	1.41678485887528\\
260	2.11721947759961\\
261	3.16132808446125\\
262	-0.349585303420531\\
263	0.0985759247937355\\
264	2.08447917925657\\
265	-2.1555167751627\\
266	2.80128469420437\\
267	-0.967359051818671\\
268	2.29582785717839\\
269	0.23671410618897\\
270	-0.696441679463378\\
271	1.29491848926979\\
272	3.77172736298989\\
273	-1.46781384249468\\
274	-0.220996933774271\\
275	-1.74828019283562\\
276	1.51627083300474\\
277	3.49717275481901\\
278	1.53400488557615\\
279	0.756721834148443\\
280	2.01445202446078\\
281	5.54628265323231\\
282	1.28718810251151\\
283	3.29673722508699\\
284	2.50990476878681\\
285	-1.23918585318978\\
286	3.48716908900588\\
287	-0.999189440898223\\
288	3.47216316339936\\
289	-0.277382279806842\\
290	6.46716120053199\\
291	1.42623698691293\\
292	-1.06603287376714\\
293	2.88303928380507\\
294	-0.438349801543782\\
295	2.31847078805063\\
296	-0.015193517352382\\
297	-4.76292014647391\\
298	-1.29565753708233\\
299	-0.838852404318613\\
300	2.671150594052\\
301	1.14614298860346\\
302	2.14568834639046\\
303	2.86067852512292\\
304	1.18479097369174\\
305	1.64432450022461\\
306	0.403812096231967\\
307	1.37241936512146\\
308	-1.0602986097938\\
309	-1.54484599661455\\
310	0.147888883415129\\
311	-3.32256985430162\\
312	-2.08893139537836\\
313	-1.338931267786\\
314	-0.117563083624111\\
315	1.66561366625901\\
316	2.65334258333848\\
317	0.991059808238616\\
318	-0.513030663041879\\
319	3.99742285326332\\
320	-1.50530421724035\\
321	1.54651106866815\\
322	1.33287342360073\\
323	1.62560124416231\\
324	-3.12576286256617\\
325	2.16696587975893\\
326	3.97242393401425\\
327	5.99333518299575\\
328	1.05061404188095\\
329	7.3283447482594\\
330	4.29293312213804\\
331	3.1165890647372\\
332	1.18933774524485\\
333	9.26708967529929\\
334	5.11394149716744\\
335	1.15304649558534\\
336	4.47943826761441\\
337	5.06856428164281\\
338	-168.635963588646\\
339	-171.895971898358\\
340	-174.109595227224\\
341	-175.035017557869\\
342	0.751463228817768\\
343	-3.46680466655207\\
344	175.705892398526\\
345	173.483635710549\\
346	3.35913463169305\\
347	5.35559316311566\\
348	10.2947479849884\\
349	8.62977326158186\\
350	3.75526020378781\\
351	6.70756891827763\\
352	4.38212642807512\\
353	5.20439796121415\\
354	4.26803025242417\\
355	2.57211562585116\\
356	3.42346156673964\\
357	6.67527919897639\\
358	4.73253861127205\\
359	3.5902317343829\\
360	1.36976583662891\\
361	2.13294193273875\\
362	0.110678326311911\\
363	-3.11887504367503\\
364	0.379307298480665\\
365	3.21199741593217\\
366	-0.269829931534233\\
367	4.05694624433269\\
368	7.31103448758165\\
369	6.38099025330264\\
370	3.00498825601083\\
371	3.50725902323241\\
372	16.0099839050774\\
373	12.7667966437438\\
374	6.80312881429481\\
375	4.7002578565941\\
376	3.72082537820037\\
377	-3.75636626754653\\
378	-7.31223233200734\\
379	-10.8662845543567\\
380	-13.975308117371\\
381	-1.23428924788512\\
382	2.39528544333412\\
383	4.58165973250647\\
384	3.30666277061148\\
385	-0.185610830922133\\
386	-0.721970685425064\\
387	-0.453335983634731\\
388	-1.24287767715788\\
389	3.25484980934141\\
390	5.77302999263793\\
391	-1.69833667568572\\
392	1.79711839327873\\
393	4.55166324921906\\
394	1.55120906487836\\
395	1.26803038238498\\
396	1.2830289441938\\
397	-0.0174215297287148\\
398	-2.57469188782873\\
399	-1.37832801236205\\
400	0.518029721184093\\
401	-1.75015379910232\\
402	-4.55925158938695\\
403	9.91983634012542\\
404	7.11255473982978\\
405	4.05208869917234\\
406	2.92529190122783\\
407	-1.09607482050121\\
408	-5.83521673059678\\
409	6.34840043256475\\
410	-3.23207859897883\\
411	-5.49753772756005\\
412	0.969269156908748\\
413	0.921070753091789\\
414	2.91743312825749\\
415	2.63696805857702\\
416	3.63787742648562\\
417	1.45108137313203\\
418	3.7592823314353\\
419	5.88568414460035\\
420	3.6297734070354\\
421	5.90750830654584\\
422	3.25071370291215\\
423	0.27299140198545\\
424	1.59709737065197\\
425	4.90210659515408\\
426	3.38619520085441\\
427	1.74802800944486\\
428	9.09894407831191\\
429	3.96667035341216\\
430	4.72757861327415\\
431	3.50530430846932\\
432	3.32756912588725\\
433	-2.94379168203822\\
434	1.59711199152566\\
435	-0.954699509635464\\
436	1.12301682184744\\
437	1.61210948611742\\
438	3.62574359542509\\
439	2.60529262941427\\
440	1.84120363583317\\
441	3.09029475742335\\
442	2.61438190365226\\
443	0.389832006341846\\
444	-0.160159811846946\\
445	2.61347306160583\\
446	2.63346972106151\\
447	1.63574194064567\\
448	-1.42833887090829\\
449	3.3243877518379\\
450	1.29211896548373\\
};
\end{axis}
\end{tikzpicture}%
}
  \caption{The angular error of the vehicle controlled via PID control in degrees.}
  \label{fig:centerline_pid_error}
\end{figure}

\noindent Sample results can be found in the following video sequences:\\
\url{https://www.youtube.com/watch?v=w3Wnw5SLmss}\\
\url{https://youtu.be/937OZez1iN8?t=142}\\\\
Notions / resources / tools involved: $\{$PID, Predictive$\}$ Control, ROS (Linux), Python, git, MATLAB.
