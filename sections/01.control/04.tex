\subsection{Balancing a segway}

Context: Systems and Control in Practice EL2222, Royal Institute of
Technology (KTH), Stockholm, Sweden.\\

Problem statement: suppose a two-wheeled motor-led contraption equipped with a
computing unit and an IMU. The goal is to balance the system in an upright
position using information from the gyro and accelerometer. A basic solution
consists of first integrating and fusing the angular velocity and linear
acceleration measurements to a filter that estimates the system's angular error
with respect to the vertical. The second step employs a (in this case PID)
controller that acts in a way that keeps this error at zero. Figures
\ref{fig:shellfie} and \ref{fig:shellfie_schem} show the real and idealized
``shellfie" segway.


\noindent\makebox[\linewidth][c]{%
\begin{minipage}{\linewidth}
  \begin{minipage}{0.45\linewidth}
    \begin{figure}[H]\centering
      \includegraphics[scale=0.25]{images/shellfie.png}
      \caption{The custom-built ``shellfie" segway. Image courtesy of Jatesada ``Nicky" Borsub.}
      \label{fig:shellfie}
    \end{figure}
  \end{minipage}
  \hfill
  \begin{minipage}{0.45\linewidth}
    \begin{figure}[H]\centering
      \includegraphics[scale=0.32]{images/shellfie_schem.png}
      \caption{The shellfie is in principle an inverted pendulum.}
      \label{fig:shellfie_schem}
    \end{figure}
  \end{minipage}
\end{minipage}
}
