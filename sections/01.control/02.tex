\subsection{Tracking the circumference of a circle with a RC car}

Context: Automatic Control Project Course EL2425, KTH Royal Institute of
Technology, Stockholm, Sweden.\\

\noindent \href{https://github.com/li9i/HT16_P2_EL2425}{\texttt{[Code]}} \href{https://github.com/li9i/HT16_P2_EL2425_resources}{\texttt{[Wiki/Resources]}} \href{https://www.youtube.com/watch?v=Vh1huYlyD_8}{\texttt{[Video 1}} $|$ \href{https://youtu.be/937OZez1iN8?t=69}{\texttt{Video 2]}}


\begin{problem}
Assume a remotely controlled vehicle of Ackermann steering whose pose
$(x,y,\theta)$ is measureable in 2D space, equipped with a computing unit.
Given a circular path of appropriate radiue, the goal is for the vehicle to
navigate the path as closely as possible.
\end{problem}

The solution was sought after using Model Predictive Control, with variable
reference poses within the horizon of the optimization problem. Figure
\ref{fig:circular_mpc_trajectory_transient} shows the trajectory of the vehicle
from a top view during the transient phase; figure
\ref{fig:circular_mpc_trajectory_ss} depicts it during the steady-state phase.
Accordingly, figures \ref{fig:circular_mpc_error_transient} and
\ref{fig:circular_mpc_error_ss} show the respective errors in each phase as a
function of time. These refer to actual experiments, not simulations.  Notably,
the steady-state error does not exceed $3.5$cm, that is, the center of gravity
of the vehicle does not diverge more than $3.5$cm from the reference circular
trajectory.

\noindent\makebox[\linewidth][c]{%
\begin{minipage}{\linewidth}
  \begin{minipage}{0.45\linewidth}
    \begin{figure}[H]
      \scalebox{0.6}{\input{images/circumference_trajectory_transient.tex}}
      \caption{\small Reference trajectory (red) and trajectory of the vehicle
               (blue), in the transient phase}
      \label{fig:circular_mpc_trajectory_transient}
    \end{figure}
  \end{minipage}
  \hfill
  \begin{minipage}{0.45\linewidth}
    \begin{figure}[H]
      \scalebox{0.6}{\input{images/circumference_error_transient.tex}}
      \caption{\small The discrepancy in distance between the trajectory of the
               vehicle and the reference trajectory in the transient phase,
               in meters}
      \label{fig:circular_mpc_error_transient}
    \end{figure}
  \end{minipage}
\end{minipage}
}

\noindent\makebox[\linewidth][c]{%
\begin{minipage}{\linewidth}
  \begin{minipage}{0.45\linewidth}
    \begin{figure}[H]
      \scalebox{0.6}{% This file was created by matlab2tikz.
%
%The latest updates can be retrieved from
%  http://www.mathworks.com/matlabcentral/fileexchange/22022-matlab2tikz-matlab2tikz
%where you can also make suggestions and rate matlab2tikz.
%
\definecolor{mycolor1}{rgb}{0.00000,0.44700,0.74100}%
\definecolor{mycolor2}{rgb}{0.85000,0.32500,0.09800}%
%
\begin{tikzpicture}

\begin{axis}[%
width=4.133in,
height=3.26in,
at={(0.693in,0.44in)},
scale only axis,
xmin=-1.26881388804908,
xmax=3.16881392381186,
xlabel={meters},
ymin=-2,
ymax=1.5,
ylabel={meters},
axis background/.style={fill=white}
]
\addplot [color=mycolor1,solid,forget plot]
  table[row sep=crcr]{%
1.75968027114868	1.14259135723114\\
1.56494045257568	1.24403214454651\\
1.35722923278809	1.31764709949493\\
1.1394921541214	1.36079788208008\\
0.918106734752655	1.37017345428467\\
0.69622129201889	1.34807813167572\\
0.479987770318985	1.29257988929749\\
0.275184959173203	1.2069286108017\\
0.085549108684063	1.0919086933136\\
-0.0859459936618805	0.950152814388275\\
-0.231851190328598	0.784706771373749\\
-0.359755516052246	0.584181129932404\\
-0.440733730792999	0.39856281876564\\
-0.499197334051132	0.186272963881493\\
-0.527667760848999	-0.0320731773972511\\
-0.522507071495056	-0.252285718917847\\
-0.485003590583801	-0.469813019037247\\
-0.416174024343491	-0.681056141853333\\
-0.317623257637024	-0.878363013267517\\
-0.189981177449226	-1.05785012245178\\
-0.036351639777422	-1.21566998958588\\
0.141134232282639	-1.34934842586517\\
0.336750507354736	-1.45563530921936\\
0.545290470123291	-1.53280818462372\\
0.762614846229553	-1.57518458366394\\
0.983980655670166	-1.58440136909485\\
1.20645582675934	-1.56153798103333\\
1.42052185535431	-1.50575196743011\\
1.62508153915405	-1.41876327991486\\
1.81483423709869	-1.30409204959869\\
1.98668682575226	-1.16229510307312\\
2.13355922698975	-0.996570765972137\\
2.25527429580688	-0.811218559741974\\
2.3455126285553	-0.60851776599884\\
2.40580868721008	-0.395386666059494\\
2.43486738204956	-0.174863129854202\\
2.43091177940369	0.0464431531727314\\
2.39342999458313	0.262731999158859\\
2.32490682601929	0.471570879220963\\
2.22634768486023	0.669028043746948\\
2.10017228126526	0.848884522914886\\
1.94889199733734	1.00809061527252\\
1.77563333511353	1.14403653144836\\
1.58340501785278	1.25234711170197\\
1.37727653980255	1.33202481269836\\
1.16098189353943	1.38049364089966\\
0.939607799053192	1.3959196805954\\
0.717277050018311	1.37841784954071\\
0.500183701515198	1.32777225971222\\
0.293653547763824	1.24585795402527\\
0.102397136390209	1.13360333442688\\
-0.0705949887633324	0.995602190494537\\
-0.219802439212799	0.831059157848358\\
-0.342859596014023	0.649230539798737\\
-0.437286883592606	0.450291752815247\\
-0.502143979072571	0.238313123583794\\
-0.536170303821564	0.0205320063978434\\
-0.538541436195374	-0.200819611549377\\
-0.508017003536224	-0.419782102108002\\
-0.445630103349686	-0.633374691009521\\
-0.354497969150543	-0.834356129169464\\
-0.232914417982101	-1.0196944475174\\
-0.0859327614307404	-1.18424129486084\\
0.0855725109577179	-1.32378494739532\\
0.277695417404175	-1.43832194805145\\
0.483262062072754	-1.5229469537735\\
0.699066400527954	-1.57577693462372\\
0.92104184627533	-1.59555792808533\\
1.14411652088165	-1.58106780052185\\
1.36079287528992	-1.53461575508118\\
1.56950795650482	-1.45577657222748\\
1.76372313499451	-1.34822762012482\\
1.94129252433777	-1.21300864219666\\
2.09556984901428	-1.0515820980072\\
2.22432637214661	-0.872050940990448\\
2.32435774803162	-0.674217700958252\\
2.39376449584961	-0.462617158889771\\
2.43241500854492	-0.244190961122513\\
2.4389636516571	-0.0217414125800133\\
2.41253137588501	0.196615859866142\\
2.35454750061035	0.408983290195465\\
2.26483845710754	0.611000299453735\\
2.14867472648621	0.799187004566193\\
2.00496411323547	0.964099407196045\\
1.83891534805298	1.1094765663147\\
1.65210473537445	1.22810864448547\\
1.45024406909943	1.31729936599731\\
1.23631358146667	1.37585353851318\\
1.01623427867889	1.40045893192291\\
0.796075344085693	1.38983500003815\\
0.578654646873474	1.3456494808197\\
0.371656209230423	1.26944828033447\\
0.177857056260109	1.16288864612579\\
0.000464876706246287	1.03030180931091\\
-0.155409082770348	0.872758507728577\\
-0.285812526941299	0.695455491542816\\
-0.39082944393158	0.501626133918762\\
-0.465449452400208	0.29343768954277\\
-0.50944596529007	0.0771294459700584\\
-0.523095667362213	-0.1436597853899\\
-0.502262532711029	-0.363042324781418\\
-0.450368911027908	-0.578320920467377\\
-0.367180377244949	-0.783585906028748\\
-0.254384011030197	-0.972686588764191\\
-0.114413119852543	-1.14333510398865\\
0.0514484792947769	-1.28990888595581\\
0.238039538264275	-1.41126918792725\\
0.440337240695953	-1.50363314151764\\
0.654061198234558	-1.56375122070312\\
0.873750627040863	-1.59043824672699\\
1.09684610366821	-1.58307015895844\\
1.31499576568604	-1.54243791103363\\
1.52498161792755	-1.47087490558624\\
1.72194480895996	-1.36820816993713\\
1.90227603912354	-1.23893845081329\\
2.06127715110779	-1.08264708518982\\
2.1959924697876	-0.907033622264862\\
2.30168914794922	-0.712703764438629\\
2.37663674354553	-0.504148185253143\\
2.4210832118988	-0.286991208791733\\
2.43299221992493	-0.0647144168615341\\
2.41204833984375	0.154098138213158\\
2.35899972915649	0.367271542549133\\
2.27404737472534	0.571486532688141\\
2.16164922714233	0.760901570320129\\
2.02202963829041	0.930207133293152\\
1.85881543159485	1.07779467105865\\
1.67576718330383	1.20061218738556\\
1.47600150108337	1.29488027095795\\
1.26368844509125	1.35971736907959\\
1.04444181919098	1.39213192462921\\
0.821956694126129	1.3902370929718\\
0.602169454097748	1.35502207279205\\
0.390409499406815	1.28819370269775\\
0.191285848617554	1.18907058238983\\
0.0102382712066174	1.06194615364075\\
-0.151029914617538	0.90937203168869\\
-0.285575777292252	0.734880924224854\\
-0.393245249986649	0.543407678604126\\
-0.472007215023041	0.336531430482864\\
-0.518703818321228	0.120280787348747\\
-0.534978449344635	-0.0998640507459641\\
-0.519013345241547	-0.320601999759674\\
-0.47124382853508	-0.53762674331665\\
-0.392446875572205	-0.744948625564575\\
-0.284037888050079	-0.937700629234314\\
-0.147938683629036	-1.11215126514435\\
0.0147905061021447	-1.26241683959961\\
0.198011055588722	-1.38965702056885\\
0.398804366588593	-1.48725712299347\\
0.610560655593872	-1.55281388759613\\
0.830131828784943	-1.58552277088165\\
1.05355155467987	-1.58357012271881\\
1.27339553833008	-1.5485030412674\\
1.48568964004517	-1.48220896720886\\
1.68577039241791	-1.3842316865921\\
1.8687424659729	-1.25926578044891\\
2.03289914131165	-1.10725426673889\\
2.17141819000244	-0.934117019176483\\
2.28155469894409	-0.741312444210052\\
2.36149287223816	-0.534054696559906\\
2.41017413139343	-0.317327737808228\\
2.42691802978516	-0.0957252457737923\\
2.41126465797424	0.125396937131882\\
2.36200141906738	0.338151812553406\\
2.28118705749512	0.543889880180359\\
2.17209982872009	0.734978914260864\\
2.03649210929871	0.907663345336914\\
1.87658250331879	1.05795764923096\\
1.69564616680145	1.18383800983429\\
1.4975084066391	1.28077626228333\\
1.28699684143066	1.34792077541351\\
1.06751883029938	1.38288712501526\\
0.845565855503082	1.38343060016632\\
0.625146865844727	1.35142850875854\\
0.411604166030884	1.28759741783142\\
0.211630403995514	1.19168293476105\\
0.0277855768799782	1.06714498996735\\
-0.135697901248932	0.916755676269531\\
-0.272311061620712	0.744242131710052\\
-0.38231098651886	0.554316103458405\\
-0.462709337472916	0.348116457462311\\
-0.511806845664978	0.133379340171814\\
-0.531381547451019	-0.0863088890910149\\
-0.518644273281097	-0.307263970375061\\
-0.473634570837021	-0.523561000823975\\
-0.398153960704803	-0.732131659984589\\
-0.292086005210876	-0.925232291221619\\
-0.158766895532608	-1.10092961788177\\
0.00206595240160823	-1.25234055519104\\
0.184461146593094	-1.37929129600525\\
0.384055972099304	-1.47703325748444\\
0.595630764961243	-1.54389619827271\\
0.814789295196533	-1.57769620418549\\
1.03684020042419	-1.57726490497589\\
1.25818073749542	-1.54501569271088\\
1.46947157382965	-1.47977042198181\\
1.67068791389465	-1.384122133255\\
1.85552024841309	-1.2614004611969\\
2.02056121826172	-1.11147952079773\\
2.16182804107666	-0.940569937229156\\
2.27510118484497	-0.749503314495087\\
2.35689568519592	-0.543516635894775\\
2.40879368782043	-0.327951967716217\\
2.42844891548157	-0.106428503990173\\
2.41454720497131	0.11366294324398\\
2.36753940582275	0.326685458421707\\
2.28800320625305	0.532724499702454\\
2.18017530441284	0.723637819290161\\
2.04535222053528	0.896851122379303\\
1.88579869270325	1.04773151874542\\
1.70496654510498	1.1735817193985\\
1.50755929946899	1.27118587493896\\
1.29701828956604	1.33944797515869\\
1.07832729816437	1.37531268596649\\
0.855926334857941	1.37826359272003\\
0.634816110134125	1.34781002998352\\
0.422103017568588	1.28438222408295\\
0.221329465508461	1.19017541408539\\
0.0370074957609177	1.06605637073517\\
-0.127075910568237	0.917067110538483\\
-0.265034139156342	0.745553553104401\\
-0.376079887151718	0.555877685546875\\
-0.456724613904953	0.350808471441269\\
-0.506422162055969	0.135810986161232\\
-0.525823891162872	-0.0840787291526794\\
-0.512270271778107	-0.30419597029686\\
-0.466797649860382	-0.520885705947876\\
-0.389747887849808	-0.729281067848206\\
-0.282568603754044	-0.921707212924957\\
-0.148145034909248	-1.09636151790619\\
};
\addplot [color=mycolor2,solid,forget plot]
  table[row sep=crcr]{%
1.51190984249115	1.29077577590942\\
1.31288290023804	1.3554435968399\\
1.08073365688324	1.39429199695587\\
0.819266378879547	1.39429199695587\\
0.612573444843292	1.3615550994873\\
0.38809010386467	1.29077577590942\\
0.177442893385887	1.18575096130371\\
0.00601941347122192	1.06571888923645\\
-0.164717242121696	0.90369588136673\\
-0.308005839586258	0.716958522796631\\
-0.398191064596176	0.557556748390198\\
-0.491892546415329	0.313456028699875\\
-0.531532526016235	0.134651690721512\\
-0.550000011920929	-0.100000001490116\\
-0.535402119159698	-0.308759659528732\\
-0.4844571352005	-0.538557529449463\\
-0.409461677074432	-0.733927369117737\\
-0.278728067874908	-0.960364639759064\\
-0.147030547261238	-1.12299752235413\\
0.00601941347122192	-1.26571893692017\\
0.200000002980232	-1.39903807640076\\
0.412448078393936	-1.50037062168121\\
0.638132452964783	-1.56722140312195\\
0.845365285873413	-1.59634602069855\\
1.05463469028473	-1.59634602069855\\
1.28742659091949	-1.56155514717102\\
1.51190984249115	-1.49077582359314\\
1.70000004768372	-1.39903807640076\\
1.89398062229156	-1.26571893692017\\
2.06471729278564	-1.1036958694458\\
2.19355630874634	-0.938789367675781\\
2.29819107055664	-0.757556736469269\\
2.38445711135864	-0.538557529449463\\
2.43540215492249	-0.308759659528732\\
2.45000004768372	-0.100000001490116\\
2.43153262138367	0.134651690721512\\
2.38445711135864	0.338557571172714\\
2.30946159362793	0.533927381038666\\
2.19355630874634	0.73878937959671\\
2.06471729278564	0.90369588136673\\
1.89398062229156	1.06571888923645\\
1.72255706787109	1.18575096130371\\
1.53609669208527	1.28075730800629\\
1.31288290023804	1.3554435968399\\
1.10679268836975	1.39178287982941\\
0.845365285873413	1.39634609222412\\
0.638132452964783	1.36722135543823\\
0.412448078393936	1.30037069320679\\
0.22278556227684	1.21192955970764\\
0.0265077874064445	1.08201611042023\\
-0.129009693861008	0.941987574100494\\
-0.263525485992432	0.781677901744843\\
-0.386509776115417	0.580985724925995\\
-0.468277871608734	0.388352245092392\\
-0.522440791130066	0.186213493347168\\
-0.549771547317505	-0.0738213881850243\\
-0.541782855987549	-0.256792694330215\\
-0.498888731002808	-0.488228559494019\\
-0.430757284164429	-0.686096668243408\\
-0.308005839586258	-0.916958570480347\\
-0.199066668748856	-1.06418144702911\\
-0.0141814146190882	-1.24906671047211\\
0.133041441440582	-1.35800588130951\\
0.339895039796829	-1.47031819820404\\
0.56177145242691	-1.54888868331909\\
0.793207287788391	-1.59178280830383\\
1.00234925746918	-1.59908628463745\\
1.23621344566345	-1.57244074344635\\
1.438352227211	-1.51827788352966\\
1.65420734882355	-1.42442142963409\\
1.85272252559662	-1.29795324802399\\
2.01066017150879	-1.16066014766693\\
2.16352558135986	-0.981677889823914\\
2.27442145347595	-0.804207324981689\\
2.35953903198242	-0.613030195236206\\
2.42244076728821	-0.386213481426239\\
2.44908618927002	-0.152349248528481\\
2.44178295135498	0.0567926950752735\\
2.39888882637024	0.288228571414948\\
2.34077572822571	0.461909890174866\\
2.23575091362	0.67255711555481\\
2.11571884155273	0.843980610370636\\
1.95369589328766	1.01471722126007\\
1.78878939151764	1.14355635643005\\
1.53609669208527	1.28075730800629\\
1.31288290023804	1.3554435968399\\
1.08073365688324	1.39429199695587\\
0.819266378879547	1.39429199695587\\
0.66378653049469	1.37244081497192\\
0.511442422866821	1.33445715904236\\
0.29244327545166	1.24819111824036\\
0.111210644245148	1.14355635643005\\
-0.053695909678936	1.01471722126007\\
-0.215718939900398	0.843980610370636\\
-0.361929565668106	0.627214431762695\\
-0.440775781869888	0.461909890174866\\
-0.511555075645447	0.237426578998566\\
-0.544292032718658	0.0307336132973433\\
-0.546346068382263	-0.204634711146355\\
-0.511555075645447	-0.437426567077637\\
-0.450370639562607	-0.63755190372467\\
-0.34903809428215	-0.850000023841858\\
-0.232016131281853	-1.0234922170639\\
-0.0729975402355194	-1.19703054428101\\
0.0896353423595428	-1.32872807979584\\
0.29244327545166	-1.44819104671478\\
0.536543965339661	-1.54189252853394\\
0.741240322589874	-1.58540213108063\\
0.976178586483002	-1.59977149963379\\
1.15875959396362	-1.58540213108063\\
1.41352546215057	-1.52658474445343\\
1.60755670070648	-1.44819104671478\\
1.81036460399628	-1.32872807979584\\
1.97299754619598	-1.19703054428101\\
2.1320161819458	-1.0234922170639\\
2.26192951202393	-0.827214419841766\\
2.34077572822571	-0.661909878253937\\
2.41155505180359	-0.437426567077637\\
2.44634604454041	-0.204634711146355\\
2.44429206848145	0.0307336132973433\\
2.41155505180359	0.237426578998566\\
2.35037064552307	0.437551915645599\\
2.26192951202393	0.627214431762695\\
2.1320161819458	0.823492228984833\\
1.99198758602142	0.979009687900543\\
1.81036460399628	1.12872803211212\\
1.63098573684692	1.23650979995728\\
1.41352546215057	1.326584815979\\
1.21047222614288	1.37721157073975\\
0.949999988079071	1.39999997615814\\
0.741240322589874	1.38540208339691\\
0.511442422866821	1.33445715904236\\
0.31607261300087	1.25946164131165\\
0.111210644245148	1.14355635643005\\
-0.053695909678936	1.01471722126007\\
-0.199066668748856	0.864181399345398\\
-0.335750937461853	0.67255711555481\\
-0.430757284164429	0.486096680164337\\
-0.498888731002808	0.288228571414948\\
-0.541782855987549	0.0567926950752735\\
-0.549086213111877	-0.152349248528481\\
-0.522440791130066	-0.386213481426239\\
-0.45953893661499	-0.613030195236206\\
-0.361929565668106	-0.827214419841766\\
-0.232016131281853	-1.0234922170639\\
-0.110660172998905	-1.16066014766693\\
0.0683221220970154	-1.31352543830872\\
0.245792657136917	-1.42442142963409\\
0.486474514007568	-1.52658474445343\\
0.689527750015259	-1.57721161842346\\
0.92382138967514	-1.59977149963379\\
1.13280403614044	-1.5888192653656\\
1.36345601081848	-1.54189252853394\\
1.58392739295959	-1.45946168899536\\
1.7669585943222	-1.35800588130951\\
1.95369589328766	-1.21471726894379\\
2.09906673431396	-1.06418144702911\\
2.23575091362	-0.872557103633881\\
2.34077572822571	-0.661909878253937\\
2.40544366836548	-0.462882846593857\\
2.44429206848145	-0.230733618140221\\
2.44634604454041	0.00463471049442887\\
2.41722130775452	0.211867541074753\\
2.35037064552307	0.437551915645599\\
2.27442145347595	0.604207336902618\\
2.14795327186584	0.80272251367569\\
2.01066017150879	0.96066015958786\\
1.83167791366577	1.11352550983429\\
1.65420734882355	1.22442138195038\\
1.438352227211	1.31827783584595\\
1.23621344566345	1.37244081497192\\
1.00234925746918	1.39908623695374\\
0.767195999622345	1.38881921768188\\
0.536543965339661	1.34189260005951\\
0.31607261300087	1.25946164131165\\
0.133041441440582	1.15800583362579\\
-0.053695909678936	1.01471722126007\\
-0.199066668748856	0.864181399345398\\
-0.335750937461853	0.67255711555481\\
-0.420318186283112	0.510104954242706\\
-0.498888731002808	0.288228571414948\\
-0.538819253444672	0.0828040167689323\\
-0.549086213111877	-0.152349248528481\\
-0.527211606502533	-0.36047226190567\\
-0.468277871608734	-0.588352203369141\\
-0.374421387910843	-0.804207324981689\\
-0.24795326590538	-1.00272250175476\\
-0.110660172998905	-1.16066014766693\\
0.0472774654626846	-1.29795324802399\\
0.245792657136917	-1.42442142963409\\
0.461647778749466	-1.51827788352966\\
0.689527750015259	-1.57721161842346\\
0.89765077829361	-1.59908628463745\\
1.13280403614044	-1.5888192653656\\
1.33822858333588	-1.54888868331909\\
1.56010496616364	-1.47031819820404\\
1.74487888813019	-1.37207210063934\\
1.93408858776093	-1.23206436634064\\
2.09906673431396	-1.06418144702911\\
2.23575091362	-0.872557103633881\\
2.33075737953186	-0.686096668243408\\
2.40544366836548	-0.462882846593857\\
2.44429206848145	-0.230733618140221\\
2.4479444026947	-0.0214960649609566\\
2.41722130775452	0.211867541074753\\
2.36827778816223	0.388352245092392\\
2.27442145347595	0.604207336902618\\
2.16352558135986	0.781677901744843\\
2.01066017150879	0.96066015958786\\
1.85272252559662	1.09795331954956\\
1.65420734882355	1.22442138195038\\
1.46303021907806	1.30953896045685\\
1.23621344566345	1.37244081497192\\
1.02850389480591	1.39794433116913\\
0.767195999622345	1.38881921768188\\
0.536543965339661	1.34189260005951\\
0.339895039796829	1.27031815052032\\
0.133041441440582	1.15800583362579\\
-0.0340885445475578	1.03206431865692\\
-0.199066668748856	0.864181399345398\\
-0.335750937461853	0.67255711555481\\
-0.420318186283112	0.510104954242706\\
-0.498888731002808	0.288228571414948\\
-0.538819253444672	0.0828040167689323\\
-0.549086213111877	-0.152349248528481\\
-0.527211606502533	-0.36047226190567\\
-0.468277871608734	-0.588352203369141\\
-0.374421387910843	-0.804207324981689\\
-0.24795326590538	-1.00272250175476\\
-0.110660172998905	-1.16066014766693\\
0.0472774654626846	-1.29795324802399\\
};
\end{axis}
\end{tikzpicture}%
}
      \caption{\small Reference trajectory (red) and trajectory of the vehicle
               (blue), in steady state}
      \label{fig:circular_mpc_trajectory_ss}
    \end{figure}
  \end{minipage}
  \hfill
  \begin{minipage}{0.45\linewidth}
    \begin{figure}[H]
      \scalebox{0.6}{% This file was created by matlab2tikz.
%
%The latest updates can be retrieved from
%  http://www.mathworks.com/matlabcentral/fileexchange/22022-matlab2tikz-matlab2tikz
%where you can also make suggestions and rate matlab2tikz.
%
\definecolor{mycolor1}{rgb}{0.00000,0.44700,0.74100}%
%
\begin{tikzpicture}

\begin{axis}[%
width=4.133in,
height=3.26in,
at={(0.693in,0.44in)},
scale only axis,
xmin=0,
xmax=250,
xmajorgrids,
xlabel={time [samples]},
ymin=0,
ymax=0.035,
ymajorgrids,
ylabel={distance error [meters]},
axis background/.style={fill=white}
]
\addplot [color=mycolor1,solid,forget plot]
  table[row sep=crcr]{%
1	0.0170464080149598\\
2	0.0244657816826504\\
3	0.0250322716643123\\
4	0.0288084536287228\\
5	0.0301447197435599\\
6	0.0298925479374944\\
7	0.031571213889459\\
8	0.0302179349014429\\
9	0.0276415457574005\\
10	0.0268550769762252\\
11	0.0241628752479281\\
12	0.0248221089922406\\
13	0.0237263801499888\\
14	0.0232435172014448\\
15	0.0228694866495597\\
16	0.0197956611823989\\
17	0.0215828841109874\\
18	0.0154297927378728\\
19	0.0171045896164251\\
20	0.0110739034477453\\
21	0.0165498436531877\\
22	0.0118509152290216\\
23	0.0150158230117284\\
24	0.0126105873445025\\
25	0.0143836832190725\\
26	0.0172369997309291\\
27	0.0161800817372619\\
28	0.0217903737908044\\
29	0.0187028925398924\\
30	0.0193053899435908\\
31	0.0175349794772118\\
32	0.0156535168926773\\
33	0.0173251386156446\\
34	0.014734292014621\\
35	0.0189940372945418\\
36	0.0135742880594327\\
37	0.0150097647832906\\
38	0.0120145418033782\\
39	0.0156597277142317\\
40	0.0100436915726616\\
41	0.0153367950079433\\
42	0.0081847238237242\\
43	0.0131647799228745\\
44	0.00713372030235874\\
45	0.0115402572567243\\
46	0.00538807583850811\\
47	0.0111645352327093\\
48	0.00366349321819937\\
49	0.0130937695653275\\
50	0.00262834093752909\\
51	0.0132943832200553\\
52	0.00279507973238776\\
53	0.0135513197993295\\
54	0.00622623888496137\\
55	0.0121306852887091\\
56	0.00945278270298698\\
57	0.0130397666635178\\
58	0.00868719597129941\\
59	0.0121392457597457\\
60	0.00631838297636033\\
61	0.0103069110233111\\
62	0.00390255597467399\\
63	0.00800188183970851\\
64	0.0063985193699412\\
65	0.00886829132318413\\
66	0.00485320006473724\\
67	0.0096459586417473\\
68	0.0050478127582701\\
69	0.0137134024404391\\
70	0.00774881077815155\\
71	0.014882881475271\\
72	0.0102995991953371\\
73	0.0125205137590303\\
74	0.0130755686922646\\
75	0.0114357864903448\\
76	0.013493126298498\\
77	0.0116821151656874\\
78	0.0157022297014039\\
79	0.00898400122845476\\
80	0.0143668169504703\\
81	0.00642588355020322\\
82	0.0117463539257711\\
83	0.0036083902365571\\
84	0.00665382770958505\\
85	0.00829305505101219\\
86	0.00424462185273305\\
87	0.0119320264095719\\
88	0.00341423070611129\\
89	0.0125246881648729\\
90	0.00346694967372778\\
91	0.0129436715825446\\
92	0.0137113444397113\\
93	0.0228660561147383\\
94	0.0238040678023016\\
95	0.0289633227611837\\
96	0.0309018657207819\\
97	0.0306834891446684\\
98	0.0331658136079921\\
99	0.029916371195686\\
100	0.0274046821322429\\
101	0.0250811210089786\\
102	0.0205270012371196\\
103	0.0195035090983249\\
104	0.0128222445902479\\
105	0.0146586511402593\\
106	0.00906142336096963\\
107	0.0152673256497343\\
108	0.00679839019984821\\
109	0.0130418551663956\\
110	0.00783734105875019\\
111	0.0132229172758171\\
112	0.0131761948498955\\
113	0.0148730029091581\\
114	0.0175534636110662\\
115	0.0156306967107358\\
116	0.020837137163826\\
117	0.015582862087374\\
118	0.0188094347738465\\
119	0.0178710796503145\\
120	0.0182235792040085\\
121	0.0190914284421698\\
122	0.0164485520693112\\
123	0.0179796194540936\\
124	0.0156699337566212\\
125	0.0170872768486505\\
126	0.0136930500756367\\
127	0.0152718247479607\\
128	0.0114095338648581\\
129	0.0127889181060989\\
130	0.007721800675178\\
131	0.0110296918469432\\
132	0.0048662523315244\\
133	0.012285067463391\\
134	0.00347083571268358\\
135	0.0132396357498846\\
136	0.00565974331419233\\
137	0.0142003301797304\\
138	0.00888625424219211\\
139	0.0149885458857704\\
140	0.0126137096112322\\
141	0.0192639147995249\\
142	0.0150221658322417\\
143	0.0188181698603077\\
144	0.0132460505100085\\
145	0.0138549562689327\\
146	0.00958053288072613\\
147	0.0108842263186179\\
148	0.00937208345216\\
149	0.00958961196666105\\
150	0.0112772451002198\\
151	0.00896994756976225\\
152	0.0139627381127892\\
153	0.0128217858909009\\
154	0.0191631504813074\\
155	0.0182569014909901\\
156	0.0205356221733808\\
157	0.0232408803634404\\
158	0.0211553169696535\\
159	0.0226258534231264\\
160	0.0232517811247563\\
161	0.0234015612432189\\
162	0.0266432479450701\\
163	0.023474472788402\\
164	0.0222808539448725\\
165	0.0224593802066967\\
166	0.0218155664192784\\
167	0.0217922496467115\\
168	0.0186063514006855\\
169	0.0190507472112569\\
170	0.0158113496305015\\
171	0.0175326300973357\\
172	0.0136410925704033\\
173	0.0174557371681112\\
174	0.0129170324384068\\
175	0.0161443293927599\\
176	0.0128010695480249\\
177	0.0137609896554246\\
178	0.0149259359845725\\
179	0.0129379228148737\\
180	0.0173526278337397\\
181	0.0162073634691127\\
182	0.0207356235257507\\
183	0.0197666578358753\\
184	0.0222290249782682\\
185	0.0168244441959226\\
186	0.0184939063445173\\
187	0.0114494184950896\\
188	0.0136365656839496\\
189	0.0065619316957567\\
190	0.0139503079764484\\
191	0.00953850628464023\\
192	0.0143224035533677\\
193	0.0143465595001058\\
194	0.0171891344827632\\
195	0.0222964301751143\\
196	0.0225096861368335\\
197	0.0274092693391053\\
198	0.0285630607412954\\
199	0.0273495398097827\\
200	0.0288667701648405\\
201	0.0260277100403766\\
202	0.0244540408562652\\
203	0.0280040339910838\\
204	0.0237052832654751\\
205	0.0224894399101512\\
206	0.0214235628919951\\
207	0.0206677587428542\\
208	0.0214921630069158\\
209	0.0202143923479318\\
210	0.0205391095481278\\
211	0.0197608397932811\\
212	0.0213896763789717\\
213	0.020067189435404\\
214	0.0225287315164271\\
215	0.0191312965308041\\
216	0.020940674798267\\
217	0.0196926262877675\\
218	0.0186774553470405\\
219	0.0218028240037817\\
220	0.0191038690506234\\
221	0.0208172458911241\\
222	0.0201715733052612\\
223	0.0221748402448356\\
224	0.0235828025762012\\
225	0.0251370954718872\\
226	0.0260519292037031\\
227	0.0235777184318858\\
228	0.0249830282531989\\
229	0.0202539368179157\\
230	0.0203108430789914\\
231	0.0181226715209524\\
};
\end{axis}
\end{tikzpicture}%
}
      \caption{\small The discrepancy in distance between the trajectory of the
               vehicle and the reference trajectory in steady state, in meters}
      \label{fig:circular_mpc_error_ss}
    \end{figure}
  \end{minipage}
\end{minipage}
}\\

\noindent Notions/resources/tools involved: Predictive Control, ROS kinetic, Linux, Python, git, MATLAB
